\begin{ack}\fs
  % TODO ack
行文至此,学生感慨万分。回想起最初开始课题时的不知所措,到现在初有
成果,这其中学生所学到的,不仅是专业知识与实践技能,更对我热爱的专业,
对于科研有了全新的认识与领悟。转瞬间,在同济求学的三年时间已近尾声,成
为一名“具有科学素养的工程师”是我在入学之初记住的,今后学生也并将铭记
于心,我想这不仅是同济人的格调,更是我们看待世界的方法。

首先,我要衷心的感谢导师陈启军教授对我的教诲与启迪。我仍记得第一次
见到陈老师时,关于科研的问题,他对我的教导:任何成果都不在于冥想,不在
于他人,而在于自己实践过程的点滴。这句话为我的科研与生活指明了方向。感
谢陈老师对我知识上的教诲,科研过程中的点拨,在我迷惘无助时给予我的关心
与指引。每次与陈老师的交流都让学生受益匪浅。您的治学态度,处事原则,是
学生今后学习的楷模。

诚挚的感谢朱劲老师,张皓老师,王祝萍老师,马小峰老师,刘成菊老师,
对我的帮助与指导,是您们的传道授业解惑让学生更好的完成研究生
的学业与科研工作。

感谢张奎师兄对我的帮助与支持,感谢您给予我的每一次沟通与交流,不仅
在科研上,更在生活上给予我的关心与支持。感谢课题组的尹小川博士,王香伟
博士在课题上给予我的帮助,不厌其烦地解疑答惑,给我的论文提出了宝贵的意
见。更要感谢与我的同门与挚友王继民、孙旭和宁静,你们的帮助与陪伴是我收
获的宝贵财富。感谢实验室的熊峰博士,杜明晓博士以及给予我帮助的师弟师妹,
让我在同济的生涯永远难忘。

最后,感谢我敬爱的父亲、母亲,你们的爱与陪伴是我永远的港
湾和前行的动力。

感谢所有给予我帮助的人,感谢同济给予我的美好时光。

\ackdate
\end{ack}

%%% Local Variables:
%%% TeX-master: "../thesis.tex"
%%% End: