% 定义中英文摘要和关键字
\begin{cabstract}
在实际工程结构的服役过程中,由于非线性与随机性的耦合作用,工程结构
特别是混凝土结构的非线性反应具有不可精确预测的性质。因此,从概率密度演
化的角度考察工程结构的非线性性状是准确把握结构非线性性能的必由之路。本
文基于随机结构反应概率密度演化的思想对于随机结构分析理论进行了深入的
探讨,初步建立了随机结构反应概率密度演化的基本图景。

结构静力非线性分析是评价结构抗震性能的重要手段。对于具有双线型广义
随机本构关系材料的结构,其塑性截面分布状态的演化过程即非线性损伤构形状
态转移过程反映了结构内力演化的性质。无记忆特性结构的非线性损伤构形状态
转移过程具有马尔可夫性,通过结构的力学分析可建立风险率函数与状态转移速
率之间的关系,进一步考虑状态之间的逻辑关系,即可得到概率转移速率矩阵。
对于有记忆特性结构及力-状态联合演化过程,可通过引入相应的记忆变量构造
向量马尔可夫过程,并采用次序分析方法建立其确定性的概率密度演化方程。关
于简单结构的情况进行了解析求解,并据以探讨了结构非线性构形状态演化的若
干特征,发现了在实际应用中可能具有重要意义的稳定构形现象。讨论了力-状
态的解耦问题。基于非线性构形状态本身的性质以及演化过程的规律,初步研究
了可能的简化与近似方法。

......

最后,关于进一步工作的方向进行简要的讨论。

\end{cabstract}

\ckeywords{随机结构,马尔可夫过程,非线性构形状态,差分方法}


\begin{eabstract}
In practical engineering, the structures usually exhibits strong nonlinearity
coupled with randomness of the involved parameters. This makes it almost impossible
to exactly predict nonlinear response of the structures, particularly for the concrete
structures. To tackle the difficulty, it is necessary to capture the nonlinear performance
of the structures in the sense of probability, instead of purely deterministic standpoint.
The present thesis is the result of the efforts devoted to developing the probability
density evolution method for analysis of nonlinear stochastic structures.

......

In the finality, the problems requiring further studies are discussed.
\end{eabstract}

\ekeywords{stochastic structure, Markov process, nonlinear configuration state,
difference method}

%%% Local Variables:
%%% TeX-master: "../thesis.tex"
%%% End: